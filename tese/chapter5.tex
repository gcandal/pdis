\chapter{Conclusions} \label{chap:concl}

\section*{}

Since the start of PDIS we have gathered the needed background knowledge to
understand both the PPL and VP domains, as well as analysing the state of the art
of both topics. During this period we identified some usability features that
we believe improve the end-users' productivity, as well as some shortcomings
that we wish to overcome and mistakes to avoid. Furthermore, we have found some
evidence in the VP literature
that indicate that a VPE could enhance users' productivity when developing using
a PPL.

Having all of this into account we believe we have met the objectives proposed for
this period and have planned the future work accordingly.

\section{Future Work}

In the time I'll be dedicated to accomplish what I have proposed to do, developing
a visual programming environment and gathering knowledge of what works best in
a visual representation and what doesn't, I'll be following the work plan presented
in Figure \ref{fig:workplan}.

During these nearly 5 months, the plan is to start weighting the pros and cons
of using certain PPLs and picking one of them to be used. The next step would be
picking an apropriate frontend, deciding between Blockly, a custom HTML+Javascript
solution, or other alternative. Then, I'll start by gathering the fundamental
language constructs that must be included, defining how they could be represented
in the visual language and developing the frontend and code generation for those constructs.
From this point onwards until the end, I've planned to make iterations: of choosing
new language features to integrate, adapt the frontend and code generation for them,
converting textual examples to the visual representation and studying the differences
between the two representations. Finnally, there
will be some time to make a tutorial on how to use the VPE.

\begin{figure}[t]
  \begin{center}
    \leavevmode
    \includegraphics[width=0.86\textwidth]{workplan}
    \caption{Work plan for the work of this dissertation}
    \label{fig:workplan}
  \end{center}
\end{figure}
