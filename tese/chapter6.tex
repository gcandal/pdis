\chapter{Conclusions} \label{chap:concl}

\section*{}

Critical analysis: when to use or not, strengths and weaknesses.

\section{Future Work}

\subsection{An empirical study}

As discussed in Chapter \ref{chap:chap3}, an interesting approach to validate
the hypothesis that a

The way the study would work would be by compare how fast an user can define a model for a given set of problems
when using a VPL in a regualr way or through our graphical interface. We'd then
count the number of syntax and type erros done with each representation. By
selecting users who never used the given PPL, we could not only measure execution
speed but learning time.

It would also be valuable to assess, not only how fast can someone develop with either
of the alternatives, but also the quality of the output. That could be done in two
steps: starting by verifying if the program correctly models the problem and then
asking the participants in the study if they believe the model they
have developed graphically is easier to understand than its textual counterpart (and
vice-versa).
Although subjetive, we believe getting the participants' opinion
regarding the output quality could provide valuable insights in order to understand if VP can
really enhance an user's experience when using a PPL. Another method we could use
to help us make an assessment of the validity of the hypothesis would be asking
participants questions regarding usability. Even if it may seem redundant, since
we would already have the time measurements, it is a way of identifying strenghts and
weaknesses with reduced granularity.

\subsection{Function blocks}

A feature that was left out during development but could have a significant
impact in the VPE's usability, mainly during the development of large models
(scaling up) would be to be able to 
