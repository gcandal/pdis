\chapter{Conclusions} \label{chap:concl}

\section*{}

Critical analysis: when to use or not, strengths and weaknesses.

\section{Future Work}

During the work of this thesis, we have certainly identified areas where there
could be improvements. The first one is an addition to the validation process, while the
second concerns a feature that could be added; we describe both below.

Besides these two, the concepts we explore in this text could be further improved,
and even extended, by continuing to perform the example modeling loop.
Even if we feel that we have tackled the most significant
challenges, and provided a decent number of examples, there is always still more
work to be done in this area.

Using examples from a different PPL could be interesting, even if the semantics
are usually very similar, so we could access the generality of the concepts
and transformation patterns we have been studying.

Another feature that would highly value the VP tools for PP in respect to their
textual counterparts would be to include more visual feedback. This includes
not only graphics of the resulting distributions, but could even show progress
while the model is running, so the user could debug their program.

\subsection{An empirical study}

As discussed in Chapter \ref{chap:chap3}, an interesting approach to further validate
the hypothesis that a change in paradigm from traditional procedural or
object-oriented PP to visual PP makes a certain class of users more productive
would be to perform an empirical study whose participants would belong to the
target audience we defined in Section \ref{sec:audience}.

The way the study would work would be by compare how fast a user can define a model for a given set of problems
when using a VPL in a regualr way or through our graphical interface. We'd then
count the number of syntax and type errors done with each representation. By
selecting users who never used the chosen PPL, we could not only measure execution
speed but learning time.

It would also be valuable to assess, not only how fast can someone develop with either
of the alternatives, but also the quality of the output. That could be done in two
steps: starting by verifying if the program correctly models the problem and then
asking the participants in the study if they believe the model they
have developed graphically is easier to understand than its textual counterpart (and
vice-versa).
Although subjective, we believe getting the participants' opinion
regarding the output quality could provide valuable insights in order to understand if VP can
really enhance a user's experience when using a PPL. Another method we could use
to help us make an assessment of the validity of the hypothesis would be asking
participants questions regarding usability. Even if it may seem redundant, since
we would already have the time measurements, it is a way of identifying strengths and
weaknesses of the visual representation.

\subsection{Function blocks}

A feature that was left out during development but could have a significant
impact in the VPE's usability, mainly during the development of large models
(scaling up) would be to provide the user with the capability to create his own
blocks, made up of smaller blocks. This is analogous to functions in regular
programming and has been successfully applied to widely used VPE, as we have
seen in \ref{chap:sota}.
