\chapter*{Abstract}

Probabilistic programming is a way to create systems that help us make decisions
 in the face of uncertainty. Lots of everyday decisions involve judgment in
determining relevant factors that we do not directly observe. Historically, one
way to help make decisions under uncertainty has been to use a probabilistic
reasoning system.
Probabilistic reasoning combines our knowledge of a situation with the laws of
probability to determine those unobserved factors that are critical to the
decision. Typically, the way the several observations are combined is through
the usage of bayesian statistics, due to its anachronistic interpretation where
existing knowledge (priors) are combined with observations in order to gather
evidence towards competing hypothesis.

When compared to other machine learning methods (such as random forests, neural
networks or linear regression), which take homogeneous data as input (requiring
the user to separate their domain into different models), probabilistic
 programming is used to leverage the data’s original structure. Plus, it
 provides full probability distributions over both the predictions and
 parameters of the model, whereas ML methods can only give the user a certain
 degree of confidence on the predictions.

Until recently, probabilistic reasoning systems have been limited in scope, and
have been hard to apply to many real world situations. Models are communicated
using a mix of natural language, pseudo code, and mathematical formulae and
solved using special purpose, one-off inference methods. Rather than precise
specifications suitable for automatic inference, graphical models typically
serve as coarse, high-level descriptions, eliding critical aspects such as
fine-grained independence, abstraction and recursion.

Probabilistic programming is a new approach that makes probabilistic reasoning
systems easier to build and more widely applicable. A probabilistic programming
language (PPL) is a programming language designed to describe probabilistic
models, in a such a way we can say that the program itself is the model, and
then perform inference in those models. PPLs have seen recent interest from the
artificial intelligence, programming languages, cognitive science, and natural
languages communities. By empowering users with a common dialect in the form of
a programming language, rather than requiring each one of them to the non-trivial
and error-prone task of writing their own models and hand-tailored inference
algorithms for the problem at hand, it encourages exploration, since different
models require less time to setup and evaluate, and enables sharing knowledge
in the form of best practices, patterns and tools such as optimized compilers
or interpreters, debuggers, IDE’s, optimizers and profilers.

PPLs are closely related to graphical models and Bayesian networks, but are
more expressive and flexible. One can easily realize this by looking at the
re-usable components PPLs offer, being one of them the inference engine, which
can be plugged in into different models. For instances, it is easy to replace
the exact-solution traditional Bayesian networks inference, which requires time
exponential in the number of variables to run, with approximation algorithms
such as the Markov Chain Monte Carlo (MCMC) or Variational Message Passing
(VMP), which make it possible to compute large hierarchical models by resorting
to sampling and approximation. PPLs often extend from a basic language (i.e.,
they are embedded in a host language like R, Java or Scala), although some PPLs
such as WinBUGS and Stan offer a self-contained language, with no obvious origin
in another language.

There have been successful applications of visual programming among several
domains, being it education (MIT’s Scratch and Microsoft’s VPL), general-purpose
programming (NoFlo), 3D modeling (Blender) and data science (RapidMiner and Wek
 Knowledge Flow). The latter, being popular products, have shown that there is
 added value in providing a graphical representation for working with data.
 However, as of today no tool provides a graphical representation for a PPL.

DARPA, the main backer behind PPLs’ research, considers one of the main key
points of its Probabilistic Programming for Advancing Machine Learning program
to make models easier to write (reducing development time, encouraging
experimentation and reducing the level of expertise required to develop such
models). The use of visual programming is suitable for this kind of objectives,
so building upon the enormous flexibility of PPLs and the advantages of
probabilistic models, we want to take advantage of the graphical intuition given
by data visualization that data scientists are now accustomed to, and attempt
to provide model visualization by rethinking how to capture
the (usually textual) programmatic formalisms in a graphical manner.

The goal of this dissertation is thus to explore graphical representations of a
probabilistic programming language through the usage of node-based programming.
The hypothesis under consideration is that graphical representations (not to be
confused with bayesian graphical model), are more intuitive and easy to learn
that full-blown PPLs.

We intend to validate such hypothesis by ensuring that classical problems solved
in the literature by PPLs are also supported by our graphical representation,
and then compare the two regarding understandability and error prevention.
