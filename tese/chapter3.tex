\chapter{Problem statement}\label{chap:chap3}

\section*{}

As already described in Section \ref{sec:goals}, this dissertation aims to solve
the problem of PPLs having too much of a steep learning curve for someone who
is inexperienced in programming, even if that person would've enough knowledge
in statistics to leverage PPLs' power in applied machine learning. We propose
to do so by showing how we can develop or extend an existing VPL (more specifically
a VPE) so it can capture common PPL semantics so that it is user-friendly
and yet flexible enough to be able to express solutions for non-trivial problems
that could be solved via a PPL. Since the existing work joining VP with PP is almost nonexistent, the notable exception
being VIBES and WinBUGS (even if it they have their shortcomings, as described in \ref{sec:vibes}),
we have identified margin for improvement.

We will be comparing alternatives and proposing solutions to the common issues that
arise when trying to create a VPL with PP semantics. However, this does not mean that
the VPL must, or should, act as a mirror to the underlying PPL. The objective
is not to have a VPL that can express all of the language's syntax, acting as
a mirror or a complete interface for the textual format it represents, but rather
to enable the change of paradigm from object-oriented, procedural or declarative programming
to a visual one. nonetheless, if this scenario happens and we happen to be able
to fully represent a PPL, that is not a problem by itself, but just a consequence
of the choices made during the creation of the VPL.

VPEs have been successfully applied to other domains, and considering previous studies in VP that suggest they
are well suited for both limited domains (such as PP) and inexperienced programmers, it is the
author's belief that the PPL toolset would benefit from such a tool.
Therefore, we intend to validate the following hypothesis:

\begin{quote}
  ``When a user produces a probabilistic model via a graphical representation that automatically translates
  itself into executable code, instead of specifying it textually,
  he will do so in a shorter amount of time, make
  fewer errors (both during development and regarding the final solution) and
  will reach a final representation that is more understandable and thus
  easier to maintain.''
\end{quote}

\section{Validation Procedure}

Because there isn't a standard method for evaluating if a language is easier
to work with than another, the question arises on how to evaluate success.
The optimal way to do so would be to make an empirical
study. However, this requires finding a significant amount of people from the
target audience (described below) willing to participate in the study (which
is not an easy task, because it represents a rare profile) as well as having a tool mature and stable enough so that the
study's results could be considered reliable and representative of an underlying
idea (in this case, that a VPE can boost user's performance when developing
models with a PPL). The goal of this dissertation is to
assess several ways how such a tool could be built, pick one of them and develop
a prototype; is not to develop production-grade software.

An alternative to this empirical evaluation is to gather examples of probabilistic programs expressed in a
PPL (either the one who chose to serve as backend, or another with similar
capabilities) in its traditional textual form, translate them into our graphical
language, and compare the two regarding how easy would it be for someone other than
the program's author to understand it and the safety nets provided for the programmer
(runtime errors, compiler warnings or other forms of preventing errors).

As such, our validation procedure consists of a model-evaluate-extend loop where
we start by trying to model a problem in our VPL and then look at the result:
if we couldn't model the problem by lack of semantics, or the representation
lacked essential characteristics (such as type safety), we extend the VPL;
after this is done we continue to the next example. This iteration continues
until we are able to represent a significant amount of examples, so that
we can show how to extend a VPL to support a variety of PP constructs with different
semantics.

\section{Target audience}\label{sec:audience}

The resulting tool is aimed at people with knowledge in statistics
who are inexperienced programmers. This may include data scientists, researchers,
mathematicians or statisticians. In short, anyone who would apply PP to problem
solving but is not fluent in textual programming.

\section{Expected contributions}

As the result of this work, we built a VPE for PPLs that can
be extended with more PPLs, even if we are only implementing the adapter for one.
By doing so we: (1) provided a platform that enables other people to experiment
and make usability research on, (2) define a visual language that can be applied to PP in general,
(3) drawn some lessons learned while implementing such visual language, and (4)
initiate the study on the viability of applying VP to PP to enhance end-user's
productivity.
