\chapter{Problem statement}\label{chap:chap3}

\section*{}

As already described in Section \ref{sec:goals}, this dissertation aims to solve
the problem of PPLs having too much of a steep learning curve for someone who
is unexperienced in programming, even if that person would've enough knowledge
in statistics to leverage PPLs' power in applied machine learning. We propose
to do so by showing how we can develop or extend an existing VPE so it can capture common PPL's
semantics so that the final VPE is user-friendly
and yet flexible enough to be able to program solutions for non-trivial problems.

Since the existing work joining VP with PP is almost nonexistent, the notable exception
being VIBES and WinBUGS (even if it they have their shortcomings, as described in \ref{sec:vibes}),
we have identified margin for improvement.

VPEs have been successfuly
applied to other domains, and considering previous studies in VP that suggest it is
well suited for both limited domains and unexperienced programmers, it is the
author's belief that development in PPLs would also benefit from such a tool.
Therefore, we intend to validate the following hypotehsis:

\begin{quote}
  ``When a user produces a probabilistic model via a graphical representation that automatically translates
  itself into executable code, instead of specifying it textually,
  he will do so in a shorter amount of time, make
  fewer errors (both during development and regarding the final solution) and
  will reach a final representation that is more understandable and thus
  easier to maintain.''
\end{quote}

Because there isn't a standard method for evaluating if a language is easier
to work with than another, the question arises on how to evaluate success.
The optimal way to do so would be to make an empiral
study. However, this requires finding a significate amount of people from the
target audience (described below) willing to participate in the study, which
is not an easy task, as well as having a tool mature and stable enough so that the
study's results could be considered reliable and representative of an underlying
idea (in this case, that a VPE can boost user's performance when developing
models with a PPL). The goal of this dissertation is to
assess several ways how such a tool could be built, pick one of them and develop
a prototype; is is not to develop production-grade software.

An alternative to this empirical evaluation is to gather examples of probabilistic programs expressed in a
PPL (either the one who chose to serve as backend, or another with similar
capabilities) in its traditional textual form, translate them into our graphical
language, and compare the two. This is what we will be doing.

O que: VPL -> PLL: explore VPL able to capture PLL semantics, but not PLL syntax (transformação mas não "espelho") => change in paradigm

Processo: Vamos usar uma PLL exemplo para demonstrar (validation procedure, parte do problem statement) -> modelar numa VPL <=> consegui? mais exemplos : extender VPL

\section{Target audience}\label{sec:audience}

The resulting tool is aimed at people with knowledge in statistics
who are unexperienced programmers. This may include data scientists, researchers,
mathematicians or staticians. In short, anyone who would apply PP to problem
solving but is not fluent in programming.

\section{Expected contributions}

By the end of this work, we would expect to have built a VPE for PPLs that can
be extended with more PPLs, even if we are only implementing the adapter for one.
By doing so we will: have a platform that enables other people to experiment
and make usability research on, define a visual language that can be applied to PP in general,
identifying what works and what to avoid,
and ultimately assessing the viability of applying VP to PP to enhance end-user's
productivity.
