\chapter{Solution prototype}\label{chap:chap3}


\section{The problem it solves}

As already described in Section \ref{sec:goals}, this dissertation aims to solve
the problem of PPLs having too much of a steep learning curve for someone who
is unexperienced in programming, even if that person would've enough knowledge
in statistics to leverage PPLs' power in applied machine learning. We propose
to do so by developing a visual programming environment for a PPL that is user-friendly
and yet flexible enough to be able to program solutions for non-trivial problems.

Since the existing work joining VP with PP is still nonexistent, the notable exception
being VIBES (even if it has its shortcomins, as described in \ref{sec:vibes}),
we have identified margin for improvement.

VPEs have been successfuly
applied to other domains, and considering previous studies in VP that suggest it is
well suited for both limited domains and unexperienced programmers, it is the
author's belief that development in PPLs would also benefit from such a tool.
Therefore, we intend to validate the following hypotehsis:

\begin{quote}
  ``When a user instead of specifying a model textually,
  produces the model via a graphical representation that automatically translates
  itself into executable code, he will do so in a shorter amount of time, make
  fewer errors (both during development and regarding the final solution) and
  will reach a final representation that is more understandable and thus
  easier to maintain.''
\end{quote}

Because there isn't a standard method for evaluating if a language is easier
to work with than another, the question arises on how to evaluate success.

\subsection{Measuring efficiency and efficacy}

The way we're planning on validating the hypothesis is through an empirical study
where we compare how fast an user can define a model for a given set of problems
when using a VPL in a regualr way or through our graphical interface. We'll also
count the number of syntax and type erros done with each representation. By
selecting users who never used the given PPL, we can not only measure execution
speed but learning time.

It is valuable to assess, not only how fast can someone develop with either
of the alternatives, but also the quality of the output. We plan to do so in two
steps: starting by verifying if the program correctly models the problem and
asking the participants in the study if they believe the model they
have developed graphically is easier to understand than its textual counterpart (and
vice-versa).
Although subjetive, we believe getting the participants' opinion
regarding the output quality could provide valuable insights in order to understand if VP can
really enhance an user's experience when using a PPL. Another method we will use
to help us make an assessment of the validity of the hypothesis is asking
participants questions regarding usability. Even if it may seem redundant, since
we would already have the time measurements, it is a way of strenghts and
weaknesses.

\subsection{Target audience}

Both the study and the tool are aimed at people with knowledge in statistics
who are unexperienced programmers. This may include data scientists, researchers,
mathematicians or staticians. In short, anyone who would apply PP to problem
solving but are not fluent in programming.

\subsection{Expected contributions}

By the end of this work, we would expect to have built a VPE for PPLs that can
be extended with more PPLs, even if we are only implementing the adapter for one.
By doing so we will: have a platform that enables other people to experiment
and make usability research on, define a visual language that can be applied to PP in general,
identifying what works and what to avoid,
and ultimately assessing the viability of applying VP to PP to enhance end-user's
productivity.

\section{Outline}

\section{Architecture}

\section{Implementation}

 "controlled dataflow vpl"

\subsection{Picking a front-end}

\subsubsection{Blockly}

\subsubsection{GoJS}

\subsubsection{Custom implementation}

\subsection{Picking a target PPL}

\subsubsection{WebPPL}

\subsubsection{PyMC}

\subsubsection{Infer.NET}

\subsection{Defining a Grammar}

\subsection{Handling Cycles and Conditionals}

\subsection{Handling objects and arrays}

\subsection{Inverse Compilation}

\subsection{Instant visual results}

\subsection{Opening to extension}

\section{Tutorial}

\section{Conclusions}
