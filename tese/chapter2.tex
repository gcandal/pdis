\chapter{Background & State of the Art} \label{chap:sota}

\section*{}

This chapter has two purposes: describing the foundations on which this work
is built on, namely PP and ML, and enumerating

\iffalse
Neste capítulo é descrito o estado da arte e são
apresentados trabalhos relacionados para mostrar o que existe no
mesmo domínio e quais os problemas em aberto.
Deve deixar claro que existe uma oportunidade de desenvolvimento que
cobre alguma falha concreta .

O capítulo deve também efetuar uma revisão tecnológica às principais
ferramentas utilizáveis no âmbito do projeto, justificando futuras
escolhas.
\fi

\subsection{Machine learning}

Machine learning is a field with the potential to impact a wide spectrum of
different areas such as biology, medicine, finance, astronomy
\cite{Amatriain:2013:BDU:2541176.2514691}, computer vision, sales forecast,
robotics \cite{intml}, product recommendations, fraud detection or
internet ads bidding \cite{SciPy}. It can be seen as a subfield of artifical
intelligence that incorporates mathematics and statistics and is concerned
with conceiving algorithms that learn autonomously, that is, without human
interventation \cite{mlbrit}\cite{mlnot}.

\subsubsection{Bayesian Reasoning}

\subsubsection{Probabilistic Programming}

\subsection{Visual Programming}

\subsubsection{Visual Dataflow Programming}

\subsection{State of the Art}

\subsubsection{Stan}

\subsubsection{WinBUGS}

\subsubsection{Church}

\subsubsection{Infer.NET}

\subsubsection{PyMC}

\subsubsection{VIBES}

\subsubsection{NoFlo}

\subsubsection{RapidMiner}

\subsubsection{Weka Knowledge FLow}

\subsubsection{GoJS}

\subsubsection{Blockly}

\section{Conclusions}
